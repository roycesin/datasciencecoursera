\HeaderA{comGetObject}{Get Access to Existing COM Objects}{comGetObject}
\keyword{programming}{comGetObject}
\keyword{interface}{comGetObject}
\begin{Description}\relax
Tries to find a running instance of the specified \code{progid} and returns
this object.

Use this function to use services provided by an already running application,
e.g. the currently visible instance of Microsoft Excel.

The object will be released automatically when the last reference to the
object is removed. To speed up this process, assign some NULL value as soon
as you don't need the object any more.
\end{Description}
\begin{Usage}
\begin{verbatim}
comGetObject(progid)
\end{verbatim}
\end{Usage}
\begin{Arguments}
\begin{ldescription}
\item[\code{progid}] ProgId of COM object to get access to, e.g. "Excel.Application"
\end{ldescription}
\end{Arguments}
\begin{Value}
\code{comGetObject} returns an object of class \code{"COMObject"}.
\end{Value}
\begin{Author}\relax
Thomas Baier
\end{Author}
\begin{Examples}
\begin{ExampleCode}
# get access to excel
## Not run: x<-comGetObject("Excel.Application")

# and make it visible
## Not run: comSetProperty(x,"Visible",TRUE);

# do something now...
\end{ExampleCode}
\end{Examples}

