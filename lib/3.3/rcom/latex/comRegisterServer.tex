\HeaderA{comRegisterServer}{Initialize the COM server}{comRegisterServer}
\keyword{programming}{comRegisterServer}
\keyword{interface}{comRegisterServer}
\begin{Description}\relax
This function registers the COM server at runtime. It is called automatically
when the package \code{rcom} is loaded.

Technically, this will register the class factory in the system, so calls
to \code{CoCreateInstance()} from client applications will succeed.

\emph{Remark: The type library will be loaded on demand later on}

\code{comUnregisterServer()} is used to unregister the class object (class
factory) again.
\end{Description}
\begin{Usage}
\begin{verbatim}
comRegisterServer()
\end{verbatim}
\end{Usage}
\begin{Arguments}
\end{Arguments}
\begin{Author}\relax
Thomas Baier
\end{Author}
\begin{SeeAlso}\relax
\code{\LinkA{comUnregisterServer}{comUnregisterServer}},
\code{\LinkA{comRegisterRegistry}{comRegisterRegistry}}
\end{SeeAlso}

