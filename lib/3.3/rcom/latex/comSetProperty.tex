\HeaderA{comSetProperty}{Set One of the COM Object's Properties}{comSetProperty}
\keyword{programming}{comSetProperty}
\keyword{interface}{comSetProperty}
\begin{Description}\relax
Set one of the COM object's properties
\end{Description}
\begin{Usage}
\begin{verbatim}
comSetProperty(handle,property,value)
\end{verbatim}
\end{Usage}
\begin{Arguments}
\begin{ldescription}
\item[\code{handle}] COM object (class "COMObject") as returned by e.g. 
\code{comCreateObject}
\item[\code{property}] name of property to get as a character string
\item[\code{value}] the value the property should be set to
\end{ldescription}
\end{Arguments}
\begin{Author}\relax
Thomas Baier
\end{Author}
\begin{SeeAlso}\relax
\code{\LinkA{comInvoke}{comInvoke}}, \code{\LinkA{comGetProperty}{comGetProperty}}
\end{SeeAlso}
\begin{Examples}
\begin{ExampleCode}
# start up excel
## Not run: x<-comCreateObject("Excel.Application")

# and make it visible
## Not run: comSetProperty(x,"Visible",TRUE);

# do something now...
\end{ExampleCode}
\end{Examples}

