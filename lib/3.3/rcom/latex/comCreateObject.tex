\HeaderA{comCreateObject}{Create COM Objects}{comCreateObject}
\keyword{programming}{comCreateObject}
\keyword{interface}{comCreateObject}
\begin{Description}\relax
Creates a new COM object specified by its program identifier.

The object will be released automatically when the last reference to the
object is removed. To speed up this process, assign some NULL value as soon
as you don't need the object any more.
\end{Description}
\begin{Usage}
\begin{verbatim}
comCreateObject(progid)
\end{verbatim}
\end{Usage}
\begin{Arguments}
\begin{ldescription}
\item[\code{progid}] ProgId of COM object to create, e.g. "Excel.Application"
\end{ldescription}
\end{Arguments}
\begin{Value}
\code{comCreateObject} returns an object of class \code{"COMObject"}.
\end{Value}
\begin{Author}\relax
Thomas Baier
\end{Author}
\begin{Examples}
\begin{ExampleCode}
# start up excel
## Not run: x<-comCreateObject("Excel.Application")

# and make it visible
## Not run: comSetProperty(x,"Visible",TRUE)

# do something now...
\end{ExampleCode}
\end{Examples}

