\HeaderA{comGetProperty}{Read One of a COM Object's Properties}{comGetProperty}
\keyword{programming}{comGetProperty}
\keyword{interface}{comGetProperty}
\begin{Description}\relax
Reads one of the COM object's properties.
\end{Description}
\begin{Usage}
\begin{verbatim}
comGetProperty(handle,property,...)
\end{verbatim}
\end{Usage}
\begin{Arguments}
\begin{ldescription}
\item[\code{handle}] COM object (class "COMObject") as returned by e.g.
\code{comCreateObject}
\item[\code{property}] name of property to get as a character string
\item[\code{...}] optional additional arguements for property specification (e.g.
index for an array)
\end{ldescription}
\end{Arguments}
\begin{Value}
The return value depends on the (data) type of the property to read.
\end{Value}
\begin{Author}\relax
Thomas Baier
\end{Author}
\begin{SeeAlso}\relax
\code{\LinkA{comInvoke}{comInvoke}}, \code{\LinkA{comSetProperty}{comSetProperty}}
\end{SeeAlso}
\begin{Examples}
\begin{ExampleCode}
# start up excel
## Not run: x<-comCreateObject("Excel.Application")

# retrieve the "Visible" property
## Not run: v <- comGetProperty(x,"Visible")

# add a new workbook to Excel and gain access to the first worksheet
## Not run: newwb <- comInvoke(comGetProperty(x,"Workbooks"),"Add")
## Not run: ws <- comGetProperty(newwb,"Worksheets",1)

# get a specific range
## Not run: r <- comGetProperty(ws,"Range","A1","B4")

# do something now...
\end{ExampleCode}
\end{Examples}

